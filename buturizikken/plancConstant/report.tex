\documentclass[a4paper,11pt,dvipdfmx]{jsarticle}

% 数式
\usepackage{amsmath,amsfonts}
\usepackage{bm}
\usepackage{physics}
\usepackage{mathtools}
% 画像
\usepackage[dvipdfmx]{graphicx}
\usepackage{circuitikz}
\usepackage{amssymb}
\usepackage{siunitx}
\usepackage{float}
\usepackage{tikz}
\usepackage{askmaps}
\usepackage{multirow}
\usepackage{bigstrut}
\usepackage{rotating}
\usepackage{listings}
\usepackage{subcaption}
% 表
\usepackage{makecell}
% その他
\usepackage{url}
\usepackage{ascmac}
\usepackage{cases}
\usepackage{here}
\usepackage{upgreek}
\usepackage{tocloft}
\usepackage{titlesec}


\begin{document}

\maketitle

\section{要旨}
本実験では、ユーイングの装置と光てこを用いて、3種類の金属棒のヤング率を測定し、得られた値から金属の種類を特定することを目的とした。測定には中央に重りを載せて棒をたわませる方法を用い、光てこにより微小変位を精密に測定した。

実験対象となった金属棒A、B、Cのヤング率はそれぞれ \(9.55 \times 10^{10}\) [Pa]、\(2.14 \times 10^{11}\) [Pa]、\(1.26 \times 10^{11}\) [Pa] と求められた。これらの値は、文献値と照らし合わせた結果、Aが黄銅、Bが鉄(鋼)、Cが銅であると推定された。さらに、誤差率はそれぞれ -5\%、-1\%〜6\%、-3\% 程度であり、装置の構造や測定精度から考えても、本実験は比較的高精度にヤング率を求められたことが示された。

\section{実験の目的}
ユーイングの装置を用いて、金属のヤング率を測定し、その値から材料の特定を行う。

\section{実験手順}
実験指導書 pp.37--43「1. ユーイングの装置による測定」に従って実施。

\section{実験結果}

\subsection{ヤング率の式の導出}
長さ\( l \)、断面積\( S \)の棒に両側から力\( F \)を加えたとき、応力は\( \frac{F}{S} \)、ひずみは\( \frac{\Delta l}{l} \)となる。
このときヤング率\( E \)は次式で定義される:
\begin{equation}
E = \frac{F l}{S \Delta l}
\end{equation}

本実験では中央に重りを載せて角棒をたわませ、その変位からヤング率を求める。理論解析により、次の関係式が得られる:
\begin{equation}
E = \frac{W l^3}{4 a^3 b e}
\end{equation}

\subsection{光てこの原理}
光てこを用いて中間降下\( e \)を測定する。鏡の脚の高さを\( z \)、鏡と尺度の距離を\( x \)、尺度の読み取りを\( y, y' \)としたとき、
\begin{equation}
e = \frac{z (y - y')}{2x}
\end{equation}
が成り立つ。

\subsection{実験データ}
金属棒の寸法および測定条件を表\ref{tbl:specs}に示す。

\renewcommand{\arraystretch}{1.2}
\begin{table}[H]
\centering
\caption{金属棒の寸法と装置のパラメータ}
\label{tbl:specs}
\begin{tabular}{|c|c|c|c|}
\hline
金属棒 & 幅 \( b \) [m] & 厚さ \( a \) [m] & 鏡と尺度の距離 \( x \) [m] \\ \hline
A & 0.015920 & 0.004974333 & 2.36 \\ \hline
B & 0.015923 & 0.004896     & 2.42 \\ \hline
C & 0.015960 & 0.004953     & 2.40 \\ \hline
\end{tabular}
\end{table}

重りを変化させて各金属棒の変位を測定し、その平均変位から中間降下\( e \)およびヤング率\( E \)を計算した。結果を表\ref{tbl:young}に示す。

\begin{table}[H]
\centering
\caption{ヤング率の測定結果}
\label{tbl:young}
\begin{tabular}{|c|c|c|c|c|}
\hline
金属棒 & \( \Delta y \) [m] & 中間降下 \( e \) [m] & ヤング率 \( E \) [Pa] & 推定金属 \\ \hline
A & 0.0791 & 0.000503 & \( 9.55 \times 10^{10} \) & 黄銅 \\ \hline
B & 0.0379 & 0.000235 & \( 2.14 \times 10^{11} \) & 鉄(鋼) \\ \hline
C & 0.0617 & 0.000386 & \( 1.26 \times 10^{11} \) & 銅 \\ \hline
\end{tabular}
\end{table}

\section{考察}
今回の実験で得られたヤング率の値は、文献に記載されている各金属の代表的なヤング率と比較して妥当なものであった。
\begin{itemize}
  \item 黄銅:約\( 1.0 \times 10^{11} \)[Pa] に対し、Aの測定値は\( 9.55 \times 10^{10} \)[Pa]で、誤差約 -5\%
  \item 鉄(鋼):\( 2.0 \sim 2.2 \times 10^{11} \)[Pa] に対し、Bの測定値は\( 2.14 \times 10^{11} \)[Pa]で、誤差は -1\%〜+6\%
  \item 銅:約\( 1.3 \times 10^{11} \)[Pa] に対し、Cの測定値は\( 1.26 \times 10^{11} \)[Pa]で、誤差約 -3\%
\end{itemize}

以上から、実験装置の精度および測定手順に大きな問題はなく、比較的正確にヤング率を求めることができたと考えられる。ただし、装置の設置状態や目視による読み取り誤差、棒の固定状態などがさらなる精度向上の鍵となる。また、光てこの構造上、測定誤差が距離\( x \)や高さ\( z \)に大きく依存するため、これらの値の測定精度を高める工夫が今後の課題となる。

\end{document}
